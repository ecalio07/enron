
% Default to the notebook output style

    


% Inherit from the specified cell style.




    
\documentclass[11pt]{article}

    
    
    \usepackage[T1]{fontenc}
    % Nicer default font than Computer Modern for most use cases
    \usepackage{palatino}

    % Basic figure setup, for now with no caption control since it's done
    % automatically by Pandoc (which extracts ![](path) syntax from Markdown).
    \usepackage{graphicx}
    % We will generate all images so they have a width \maxwidth. This means
    % that they will get their normal width if they fit onto the page, but
    % are scaled down if they would overflow the margins.
    \makeatletter
    \def\maxwidth{\ifdim\Gin@nat@width>\linewidth\linewidth
    \else\Gin@nat@width\fi}
    \makeatother
    \let\Oldincludegraphics\includegraphics
    % Set max figure width to be 80% of text width, for now hardcoded.
    \renewcommand{\includegraphics}[1]{\Oldincludegraphics[width=.8\maxwidth]{#1}}
    % Ensure that by default, figures have no caption (until we provide a
    % proper Figure object with a Caption API and a way to capture that
    % in the conversion process - todo).
    \usepackage{caption}
    \DeclareCaptionLabelFormat{nolabel}{}
    \captionsetup{labelformat=nolabel}

    \usepackage{multicol}
    \usepackage{adjustbox} % Used to constrain images to a maximum size 
    \usepackage{xcolor} % Allow colors to be defined
    \usepackage{enumerate} % Needed for markdown enumerations to work
    \usepackage{geometry} % Used to adjust the document margins
    \usepackage{amsmath} % Equations
    \usepackage{amssymb} % Equations
    \usepackage{textcomp} % defines textquotesingle
    % Hack from http://tex.stackexchange.com/a/47451/13684:
    \AtBeginDocument{%
        \def\PYZsq{\textquotesingle}% Upright quotes in Pygmentized code
    }
    \usepackage{upquote} % Upright quotes for verbatim code
    \usepackage{eurosym} % defines \euro
    \usepackage[mathletters]{ucs} % Extended unicode (utf-8) support
    \usepackage[utf8x]{inputenc} % Allow utf-8 characters in the tex document
    \usepackage{fancyvrb} % verbatim replacement that allows latex
    \usepackage{grffile} % extends the file name processing of package graphics 
                         % to support a larger range 
    % The hyperref package gives us a pdf with properly built
    % internal navigation ('pdf bookmarks' for the table of contents,
    % internal cross-reference links, web links for URLs, etc.)
    \usepackage{hyperref}
    \usepackage{longtable} % longtable support required by pandoc >1.10
    \usepackage{booktabs}  % table support for pandoc > 1.12.2
    \usepackage[normalem]{ulem} % ulem is needed to support strikethroughs (\sout)
                                % normalem makes italics be italics, not underlines
   \usepackage{abstract}
   \renewcommand{\abstractnamefont}{\normalfont\tiny\bfseries}
   \renewcommand{\abstracttextfont}{\normalfont\tiny}    
  
  \usepackage{titling}
\setlength{\droptitle}{-6em}
  \newcommand{\subtitle}[1]{%
  \posttitle{%
    \par\end{center}
    \begin{center}\large#1\end{center}
   }%
   }

    
    
    % Colors for the hyperref package
    \definecolor{urlcolor}{rgb}{0,.145,.698}
    \definecolor{linkcolor}{rgb}{.71,0.21,0.01}
    \definecolor{citecolor}{rgb}{.12,.54,.11}

    % ANSI colors
    \definecolor{ansi-black}{HTML}{3E424D}
    \definecolor{ansi-black-intense}{HTML}{282C36}
    \definecolor{ansi-red}{HTML}{E75C58}
    \definecolor{ansi-red-intense}{HTML}{B22B31}
    \definecolor{ansi-green}{HTML}{00A250}
    \definecolor{ansi-green-intense}{HTML}{007427}
    \definecolor{ansi-yellow}{HTML}{DDB62B}
    \definecolor{ansi-yellow-intense}{HTML}{B27D12}
    \definecolor{ansi-blue}{HTML}{208FFB}
    \definecolor{ansi-blue-intense}{HTML}{0065CA}
    \definecolor{ansi-magenta}{HTML}{D160C4}
    \definecolor{ansi-magenta-intense}{HTML}{A03196}
    \definecolor{ansi-cyan}{HTML}{60C6C8}
    \definecolor{ansi-cyan-intense}{HTML}{258F8F}
    \definecolor{ansi-white}{HTML}{C5C1B4}
    \definecolor{ansi-white-intense}{HTML}{A1A6B2}

    % commands and environments needed by pandoc snippets
    % extracted from the output of `pandoc -s`
    \providecommand{\tightlist}{%
      \setlength{\itemsep}{0pt}\setlength{\parskip}{0pt}}
    \DefineVerbatimEnvironment{Highlighting}{Verbatim}{commandchars=\\\{\}}
    % Add ',fontsize=\small' for more characters per line
    \newenvironment{Shaded}{}{}
    \newcommand{\KeywordTok}[1]{\textcolor[rgb]{0.00,0.44,0.13}{\textbf{{#1}}}}
    \newcommand{\DataTypeTok}[1]{\textcolor[rgb]{0.56,0.13,0.00}{{#1}}}
    \newcommand{\DecValTok}[1]{\textcolor[rgb]{0.25,0.63,0.44}{{#1}}}
    \newcommand{\BaseNTok}[1]{\textcolor[rgb]{0.25,0.63,0.44}{{#1}}}
    \newcommand{\FloatTok}[1]{\textcolor[rgb]{0.25,0.63,0.44}{{#1}}}
    \newcommand{\CharTok}[1]{\textcolor[rgb]{0.25,0.44,0.63}{{#1}}}
    \newcommand{\StringTok}[1]{\textcolor[rgb]{0.25,0.44,0.63}{{#1}}}
    \newcommand{\CommentTok}[1]{\textcolor[rgb]{0.38,0.63,0.69}{\textit{{#1}}}}
    \newcommand{\OtherTok}[1]{\textcolor[rgb]{0.00,0.44,0.13}{{#1}}}
    \newcommand{\AlertTok}[1]{\textcolor[rgb]{1.00,0.00,0.00}{\textbf{{#1}}}}
    \newcommand{\FunctionTok}[1]{\textcolor[rgb]{0.02,0.16,0.49}{{#1}}}
    \newcommand{\RegionMarkerTok}[1]{{#1}}
    \newcommand{\ErrorTok}[1]{\textcolor[rgb]{1.00,0.00,0.00}{\textbf{{#1}}}}
    \newcommand{\NormalTok}[1]{{#1}}
    
    % Additional commands for more recent versions of Pandoc
    \newcommand{\ConstantTok}[1]{\textcolor[rgb]{0.53,0.00,0.00}{{#1}}}
    \newcommand{\SpecialCharTok}[1]{\textcolor[rgb]{0.25,0.44,0.63}{{#1}}}
    \newcommand{\VerbatimStringTok}[1]{\textcolor[rgb]{0.25,0.44,0.63}{{#1}}}
    \newcommand{\SpecialStringTok}[1]{\textcolor[rgb]{0.73,0.40,0.53}{{#1}}}
    \newcommand{\ImportTok}[1]{{#1}}
    \newcommand{\DocumentationTok}[1]{\textcolor[rgb]{0.73,0.13,0.13}{\textit{{#1}}}}
    \newcommand{\AnnotationTok}[1]{\textcolor[rgb]{0.38,0.63,0.69}{\textbf{\textit{{#1}}}}}
    \newcommand{\CommentVarTok}[1]{\textcolor[rgb]{0.38,0.63,0.69}{\textbf{\textit{{#1}}}}}
    \newcommand{\VariableTok}[1]{\textcolor[rgb]{0.10,0.09,0.49}{{#1}}}
    \newcommand{\ControlFlowTok}[1]{\textcolor[rgb]{0.00,0.44,0.13}{\textbf{{#1}}}}
    \newcommand{\OperatorTok}[1]{\textcolor[rgb]{0.40,0.40,0.40}{{#1}}}
    \newcommand{\BuiltInTok}[1]{{#1}}
    \newcommand{\ExtensionTok}[1]{{#1}}
    \newcommand{\PreprocessorTok}[1]{\textcolor[rgb]{0.74,0.48,0.00}{{#1}}}
    \newcommand{\AttributeTok}[1]{\textcolor[rgb]{0.49,0.56,0.16}{{#1}}}
    \newcommand{\InformationTok}[1]{\textcolor[rgb]{0.38,0.63,0.69}{\textbf{\textit{{#1}}}}}
    \newcommand{\WarningTok}[1]{\textcolor[rgb]{0.38,0.63,0.69}{\textbf{\textit{{#1}}}}}
    
    
    % Define a nice break command that doesn't care if a line doesn't already
    % exist.
    \def\br{\hspace*{\fill} \\* }
    % Math Jax compatability definitions
    \def\gt{>}
    \def\lt{<}
    % Document parameters
    
\title{Fraud Detection Machine Learning on Enron Dataset}

 
    
\author{Eduardo A. R. Calió}

 \subtitle{FEEC - University of Campinas (Unicamp)}   

    % Pygments definitions
    
\makeatletter
\def\PY@reset{\let\PY@it=\relax \let\PY@bf=\relax%
    \let\PY@ul=\relax \let\PY@tc=\relax%
    \let\PY@bc=\relax \let\PY@ff=\relax}
\def\PY@tok#1{\csname PY@tok@#1\endcsname}
\def\PY@toks#1+{\ifx\relax#1\empty\else%
    \PY@tok{#1}\expandafter\PY@toks\fi}
\def\PY@do#1{\PY@bc{\PY@tc{\PY@ul{%
    \PY@it{\PY@bf{\PY@ff{#1}}}}}}}
\def\PY#1#2{\PY@reset\PY@toks#1+\relax+\PY@do{#2}}

\expandafter\def\csname PY@tok@gd\endcsname{\def\PY@tc##1{\textcolor[rgb]{0.63,0.00,0.00}{##1}}}
\expandafter\def\csname PY@tok@gu\endcsname{\let\PY@bf=\textbf\def\PY@tc##1{\textcolor[rgb]{0.50,0.00,0.50}{##1}}}
\expandafter\def\csname PY@tok@gt\endcsname{\def\PY@tc##1{\textcolor[rgb]{0.00,0.27,0.87}{##1}}}
\expandafter\def\csname PY@tok@gs\endcsname{\let\PY@bf=\textbf}
\expandafter\def\csname PY@tok@gr\endcsname{\def\PY@tc##1{\textcolor[rgb]{1.00,0.00,0.00}{##1}}}
\expandafter\def\csname PY@tok@cm\endcsname{\let\PY@it=\textit\def\PY@tc##1{\textcolor[rgb]{0.25,0.50,0.50}{##1}}}
\expandafter\def\csname PY@tok@vg\endcsname{\def\PY@tc##1{\textcolor[rgb]{0.10,0.09,0.49}{##1}}}
\expandafter\def\csname PY@tok@vi\endcsname{\def\PY@tc##1{\textcolor[rgb]{0.10,0.09,0.49}{##1}}}
\expandafter\def\csname PY@tok@mh\endcsname{\def\PY@tc##1{\textcolor[rgb]{0.40,0.40,0.40}{##1}}}
\expandafter\def\csname PY@tok@cs\endcsname{\let\PY@it=\textit\def\PY@tc##1{\textcolor[rgb]{0.25,0.50,0.50}{##1}}}
\expandafter\def\csname PY@tok@ge\endcsname{\let\PY@it=\textit}
\expandafter\def\csname PY@tok@vc\endcsname{\def\PY@tc##1{\textcolor[rgb]{0.10,0.09,0.49}{##1}}}
\expandafter\def\csname PY@tok@il\endcsname{\def\PY@tc##1{\textcolor[rgb]{0.40,0.40,0.40}{##1}}}
\expandafter\def\csname PY@tok@go\endcsname{\def\PY@tc##1{\textcolor[rgb]{0.53,0.53,0.53}{##1}}}
\expandafter\def\csname PY@tok@cp\endcsname{\def\PY@tc##1{\textcolor[rgb]{0.74,0.48,0.00}{##1}}}
\expandafter\def\csname PY@tok@gi\endcsname{\def\PY@tc##1{\textcolor[rgb]{0.00,0.63,0.00}{##1}}}
\expandafter\def\csname PY@tok@gh\endcsname{\let\PY@bf=\textbf\def\PY@tc##1{\textcolor[rgb]{0.00,0.00,0.50}{##1}}}
\expandafter\def\csname PY@tok@ni\endcsname{\let\PY@bf=\textbf\def\PY@tc##1{\textcolor[rgb]{0.60,0.60,0.60}{##1}}}
\expandafter\def\csname PY@tok@nl\endcsname{\def\PY@tc##1{\textcolor[rgb]{0.63,0.63,0.00}{##1}}}
\expandafter\def\csname PY@tok@nn\endcsname{\let\PY@bf=\textbf\def\PY@tc##1{\textcolor[rgb]{0.00,0.00,1.00}{##1}}}
\expandafter\def\csname PY@tok@no\endcsname{\def\PY@tc##1{\textcolor[rgb]{0.53,0.00,0.00}{##1}}}
\expandafter\def\csname PY@tok@na\endcsname{\def\PY@tc##1{\textcolor[rgb]{0.49,0.56,0.16}{##1}}}
\expandafter\def\csname PY@tok@nb\endcsname{\def\PY@tc##1{\textcolor[rgb]{0.00,0.50,0.00}{##1}}}
\expandafter\def\csname PY@tok@nc\endcsname{\let\PY@bf=\textbf\def\PY@tc##1{\textcolor[rgb]{0.00,0.00,1.00}{##1}}}
\expandafter\def\csname PY@tok@nd\endcsname{\def\PY@tc##1{\textcolor[rgb]{0.67,0.13,1.00}{##1}}}
\expandafter\def\csname PY@tok@ne\endcsname{\let\PY@bf=\textbf\def\PY@tc##1{\textcolor[rgb]{0.82,0.25,0.23}{##1}}}
\expandafter\def\csname PY@tok@nf\endcsname{\def\PY@tc##1{\textcolor[rgb]{0.00,0.00,1.00}{##1}}}
\expandafter\def\csname PY@tok@si\endcsname{\let\PY@bf=\textbf\def\PY@tc##1{\textcolor[rgb]{0.73,0.40,0.53}{##1}}}
\expandafter\def\csname PY@tok@s2\endcsname{\def\PY@tc##1{\textcolor[rgb]{0.73,0.13,0.13}{##1}}}
\expandafter\def\csname PY@tok@nt\endcsname{\let\PY@bf=\textbf\def\PY@tc##1{\textcolor[rgb]{0.00,0.50,0.00}{##1}}}
\expandafter\def\csname PY@tok@nv\endcsname{\def\PY@tc##1{\textcolor[rgb]{0.10,0.09,0.49}{##1}}}
\expandafter\def\csname PY@tok@s1\endcsname{\def\PY@tc##1{\textcolor[rgb]{0.73,0.13,0.13}{##1}}}
\expandafter\def\csname PY@tok@ch\endcsname{\let\PY@it=\textit\def\PY@tc##1{\textcolor[rgb]{0.25,0.50,0.50}{##1}}}
\expandafter\def\csname PY@tok@m\endcsname{\def\PY@tc##1{\textcolor[rgb]{0.40,0.40,0.40}{##1}}}
\expandafter\def\csname PY@tok@gp\endcsname{\let\PY@bf=\textbf\def\PY@tc##1{\textcolor[rgb]{0.00,0.00,0.50}{##1}}}
\expandafter\def\csname PY@tok@sh\endcsname{\def\PY@tc##1{\textcolor[rgb]{0.73,0.13,0.13}{##1}}}
\expandafter\def\csname PY@tok@ow\endcsname{\let\PY@bf=\textbf\def\PY@tc##1{\textcolor[rgb]{0.67,0.13,1.00}{##1}}}
\expandafter\def\csname PY@tok@sx\endcsname{\def\PY@tc##1{\textcolor[rgb]{0.00,0.50,0.00}{##1}}}
\expandafter\def\csname PY@tok@bp\endcsname{\def\PY@tc##1{\textcolor[rgb]{0.00,0.50,0.00}{##1}}}
\expandafter\def\csname PY@tok@c1\endcsname{\let\PY@it=\textit\def\PY@tc##1{\textcolor[rgb]{0.25,0.50,0.50}{##1}}}
\expandafter\def\csname PY@tok@o\endcsname{\def\PY@tc##1{\textcolor[rgb]{0.40,0.40,0.40}{##1}}}
\expandafter\def\csname PY@tok@kc\endcsname{\let\PY@bf=\textbf\def\PY@tc##1{\textcolor[rgb]{0.00,0.50,0.00}{##1}}}
\expandafter\def\csname PY@tok@c\endcsname{\let\PY@it=\textit\def\PY@tc##1{\textcolor[rgb]{0.25,0.50,0.50}{##1}}}
\expandafter\def\csname PY@tok@mf\endcsname{\def\PY@tc##1{\textcolor[rgb]{0.40,0.40,0.40}{##1}}}
\expandafter\def\csname PY@tok@err\endcsname{\def\PY@bc##1{\setlength{\fboxsep}{0pt}\fcolorbox[rgb]{1.00,0.00,0.00}{1,1,1}{\strut ##1}}}
\expandafter\def\csname PY@tok@mb\endcsname{\def\PY@tc##1{\textcolor[rgb]{0.40,0.40,0.40}{##1}}}
\expandafter\def\csname PY@tok@ss\endcsname{\def\PY@tc##1{\textcolor[rgb]{0.10,0.09,0.49}{##1}}}
\expandafter\def\csname PY@tok@sr\endcsname{\def\PY@tc##1{\textcolor[rgb]{0.73,0.40,0.53}{##1}}}
\expandafter\def\csname PY@tok@mo\endcsname{\def\PY@tc##1{\textcolor[rgb]{0.40,0.40,0.40}{##1}}}
\expandafter\def\csname PY@tok@kd\endcsname{\let\PY@bf=\textbf\def\PY@tc##1{\textcolor[rgb]{0.00,0.50,0.00}{##1}}}
\expandafter\def\csname PY@tok@mi\endcsname{\def\PY@tc##1{\textcolor[rgb]{0.40,0.40,0.40}{##1}}}
\expandafter\def\csname PY@tok@kn\endcsname{\let\PY@bf=\textbf\def\PY@tc##1{\textcolor[rgb]{0.00,0.50,0.00}{##1}}}
\expandafter\def\csname PY@tok@cpf\endcsname{\let\PY@it=\textit\def\PY@tc##1{\textcolor[rgb]{0.25,0.50,0.50}{##1}}}
\expandafter\def\csname PY@tok@kr\endcsname{\let\PY@bf=\textbf\def\PY@tc##1{\textcolor[rgb]{0.00,0.50,0.00}{##1}}}
\expandafter\def\csname PY@tok@s\endcsname{\def\PY@tc##1{\textcolor[rgb]{0.73,0.13,0.13}{##1}}}
\expandafter\def\csname PY@tok@kp\endcsname{\def\PY@tc##1{\textcolor[rgb]{0.00,0.50,0.00}{##1}}}
\expandafter\def\csname PY@tok@w\endcsname{\def\PY@tc##1{\textcolor[rgb]{0.73,0.73,0.73}{##1}}}
\expandafter\def\csname PY@tok@kt\endcsname{\def\PY@tc##1{\textcolor[rgb]{0.69,0.00,0.25}{##1}}}
\expandafter\def\csname PY@tok@sc\endcsname{\def\PY@tc##1{\textcolor[rgb]{0.73,0.13,0.13}{##1}}}
\expandafter\def\csname PY@tok@sb\endcsname{\def\PY@tc##1{\textcolor[rgb]{0.73,0.13,0.13}{##1}}}
\expandafter\def\csname PY@tok@k\endcsname{\let\PY@bf=\textbf\def\PY@tc##1{\textcolor[rgb]{0.00,0.50,0.00}{##1}}}
\expandafter\def\csname PY@tok@se\endcsname{\let\PY@bf=\textbf\def\PY@tc##1{\textcolor[rgb]{0.73,0.40,0.13}{##1}}}
\expandafter\def\csname PY@tok@sd\endcsname{\let\PY@it=\textit\def\PY@tc##1{\textcolor[rgb]{0.73,0.13,0.13}{##1}}}

\def\PYZbs{\char`\\}
\def\PYZus{\char`\_}
\def\PYZob{\char`\{}
\def\PYZcb{\char`\}}
\def\PYZca{\char`\^}
\def\PYZam{\char`\&}
\def\PYZlt{\char`\<}
\def\PYZgt{\char`\>}
\def\PYZsh{\char`\#}
\def\PYZpc{\char`\%}
\def\PYZdl{\char`\$}
\def\PYZhy{\char`\-}
\def\PYZsq{\char`\'}
\def\PYZdq{\char`\"}
\def\PYZti{\char`\~}
% for compatibility with earlier versions
\def\PYZat{@}
\def\PYZlb{[}
\def\PYZrb{]}
\makeatother


    % Exact colors from NB
    \definecolor{incolor}{rgb}{0.0, 0.0, 0.5}
    \definecolor{outcolor}{rgb}{0.545, 0.0, 0.0}



    
    % Prevent overflowing lines due to hard-to-break entities
    \sloppy 
    % Setup hyperref package
    \hypersetup{
      breaklinks=true,  % so long urls are correctly broken across lines
      colorlinks=true,
      urlcolor=urlcolor,
      linkcolor=linkcolor,
      citecolor=citecolor,
      }
    % Slightly bigger margins than the latex defaults
    
    \geometry{verbose,tmargin=1in,bmargin=1in,lmargin=1in,rmargin=1in}
    
    

    \begin{document}
    
    
    \maketitle

\begin{footnotesize}
\textbf{Abstract:} All through history, corruption and fraud has been present. It is known
cases in which one book author published its book in the name of a more
famous one. The fraud was detected by the use of Machine Learning. In
such contexts, it is difficult, if not impossible, to find efficient
patterns without making use of Artificial Inteligence (AI). And since
Machine Learning has become a buzzword and has proven its efficiency.
This work explores a subfield of AI called Supervised Classificaton.

This paper covers Naive Bayes, Support Vector Mahcine (SVM), and
Decision Tree supervised classification algorithms, working on a
pre-processed list of email texts based on the Enron Corporation
dataset. As such, they will predict email authors by their writting
style and content of words.

The purpose of this work is to understand complexity, strengths,
weaknessess and how to improve accuracy and performance of the
algorithms mentioned above.

It is holped this study will guide practicioners to manage these
techniques to reach their expected results.

\textbf{Keywords:} Machine Learning, Classifier, Support Vector
Machines, Gaussian Naive Bayes, Decision Trees, GridSearch.
\end{footnotesize}
\begin{multicols}{2}

\section{Introduction}\label{introduction}

Enron was founded in 1985 through the merger of an Omaha-based natural
gas pipeline company (InterNorth) and a Texas pipeline company (Houston
Natural Gas). Within a decade, Enron had evolved beyond energy trading,
become a conglomerate that was also actively involved in the area of
metals, pulps and paper, broadband assets, water plants internationally
and also traded extensively in financial markets for the same products
and services (Healy \& Palepu, 2003). This massive expansion generated
revenues of \$ 4.6 billion in 1990, to revenues of \$101 billion in
2000. That, in turn, made Enron the seventh largest company in the
United States, bigger than IBM or Sony. Enron executives landed on the
covers of business magazines, and the company was hailed as a model for
success and innovation \cite{davis2007social}.

In order to keep appeasing the investors to create a consistent
profiting situation in the company, Enron traders were pressured to
forecast high future cash flows and low discount rate on the long-term
contract with Enron. It was also common to hide the debt since high debt
levels would lower the investment grade and trigger banks to recall
money. The lack of truthfulness by management about the health of the
company, according to Kirk Hanson, the executive director of the
Markkula Center for Applied Ethics was the cause of Enron's bankruptcy
\cite{li2010case}.

At the end of 2001, the company's corruption emmerged as a great fraud
scandal and soon after this the company filed for bankruptcy. After
that, many rules and regulations were changed to be able to audit and
prenvent cases like this one.

This scandal brought into question the accounting practices and
activities of many corporations in the United States and many rules and
regulations were changed to be able to audit and prenvent cases like
this one. And many started to wonder how such a powerful business
disintegrated so quickly and how it managed to fool the regulators for
so long.

Due to this interesting case, the company's financial and emails data
became public for studies purpose. And this case became a point of
interest for machine learning analysis because it could assist in
finding solutions on how to prevent similar situations to happen.

    \section{Data Source}\label{data-source}

The original dataset is available at
https://www.cs.cmu.edu/\textasciitilde{}./enron, and it was collected
and prepared by the CALO Project, and contains financial data and text
features, that were extracted from emails comprised of 146 users with 21
features each.

For the experiments in this paper though, two other datasets will be
used as the data sources: \textbf{word\_data.pkl} corresponding to the
\textbf{features}, and \textbf{email\_authors.pkl} to the
\textbf{labels}. They were created by Katie Malone for Udacity machine
learning training course, and represent 8.000 emails per user, belonging
to two users: Chris and Sara. It is not entirely clear how data was
developed, and not easy to understand how contents are linked by simply
opening the files.

    \section{Related Works}\label{related-works}

This research is based on the instructions presented et al.
\cite{Udacity}, which covers several machine learning techniques. Degis
et al. \cite{mdegis} presents an code develoments for training
classifiers to identify authors by emails, also based on Udacity
Training Course.

    \section{Methodology}\label{methodology}

Three different classifiers will be trainned to predict autors emails by
their word content and style, using scikit-lean library \cite{scikit}.

For performing the experiments, estimators will have their parameters
tunned by the use of scikit-learn GridSearchCV library \cite{gsearch},
that considers all parameter combinations and provide outputs of their
scores. Each experiment can be executed against only a portion or full
dataset. They are represented by two running code cells (for each
algorithm). Partial dataset is intended to speed up results so as to
focus on finding the parameters values combinations, while fulll dataset
is meant to verify how well estimators perform in a larger scale.

    \subsubsection{SVM}\label{svm}

Support Vector Machine can work with classification or regression, but
it is mostly applied to classification. While Support Vectors are the
co-ordinates of individual observation, Support Vector Machine is a
frontier which best segregates two classes. It has a feature that igores
the outliers and tend to be quite robust with them. \textbf{C:} controls
the tradeoff between smooth decision boundary and classification
training points correctly. In theory, a large value of C means that you
will get more training points correctly. \textbf{gamma:} defines how far
a the influence of a single training example reaches. If gamma has a low
value, every point has a far reach. If gamma has a high value, each
training example has a close reach. High value might make the decision
boundary less linear, for it will be closer to training points.
\textbf{kernel} parameter can be `linear', `poly', `rbf', `sigmoid',
`precomputed' or a callable. If none is given, `rbf' will be used.
GridSearch will make testes making combinations with different parameter
and values: kernel: rbf and linear gamma: 1e-3, 1e-4, C: 1, 10, 100,
1000\\

    

No trained emails for Chris: 7936\\
No trained emails for Sara: 7884
\begin{Verbatim}[commandchars=\\\{\}]
SVM with GridSearchCV and Reduced Dataset: 

Training time: 0.495 s 
Predicting time: 0.057 s

No Predicted emails for Chris: 1012 
No Predicted emails for Sara: 746

    \end{Verbatim}

    \begin{center}
    \adjustimage{max size={0.9\linewidth}{0.9\paperheight}}{Enron_Paper_GridSearch-12-06-17_files/Enron_Paper_GridSearch-12-06-17_5_1.png}
    \end{center}
    { \hspace*{\fill} \\}
    
    \begin{Verbatim}[commandchars=\\\{\}]
Best Param: \{'kernel': 'linear', 
'C': 10\}
Best Avarage Score: 0.917721518987

    \end{Verbatim}

    \subsubsection{GaussianNB (Naive Bayes)}\label{gaussiannb-naive-bayes}

Naive Bayes,based on Bayes' Theorem, works with the assumption of
independence among predictors. In simple terms, a Naive Bayes classifier
assumes that the presence of a particular feature in a class is
unrelated to the presence of any other feature. It uses Posterior
Probability, giving the rank occurance provided text. In order words, it
will be trained with frequent texts(features) used by Chris and
Sarah(labels), and it will calculate the probabily and determine if each
test email is from Chris or Sara.

It is possible to work with parameters, similarly to SVM, but parameter
tune for this classifier makes no changes in results. As such,
GridSearchCV is not applied to this scenario.


    \begin{Verbatim}[commandchars=\\\{\}]

No trained emails for Chris: 7936
No trained emails for Sara: 7884

GaussianNB with Reduced Dataset:

Training time: 0.002 s
Predicting time: 0.008 s

Total Accuracy: 0.943117178612

    \end{Verbatim}

    \subsubsection{Decision Trees}\label{decision-trees}

Decision trees work with classification or regression models, and it
breaks down a dataset into smaller subsets while at the same time an
associated decision tree is incrementally developed. The final outcome
is a tree with decision nodes and leaf nodes.

GridSearchCV will be instatiate with the tree estimator, the parameters
below and will make the predictions: parameters = \{``criterion'':
{[}``gini'', ``entropy''{]},``min\_samples\_split'': {[}2, 10,
20{]},``max\_depth'': {[}None, 2, 5, 10{]}, ``min\_samples\_leaf'':
{[}1, 5, 10{]},``max\_leaf\_nodes'': {[}None, 5, 10, 20{]}, \}


    \begin{Verbatim}[commandchars=\\\{\}]

Number of trained emails for Chris: 7936
Number of trained emails for Sara: 7884

Decision Tree with GridSearchCV
and Reduced Dataset:

Training time: 3.648 s
Predicting time: 0.001 s

Number of Predicted emails for Chris 1017
Number of Predicted emails for Sara 741

    \end{Verbatim}

    \begin{center}
    \adjustimage{max size={0.9\linewidth}{0.9\paperheight}}{Enron_Paper_GridSearch-12-06-17_files/Enron_Paper_GridSearch-12-06-17_11_1.png}
    \end{center}
    { \hspace*{\fill} \\}
    
    \begin{Verbatim}[commandchars=\\\{\}]
Best Param: \{'min\_samples\_split': 20,  
'max\_leaf\_nodes': None, 'criterion': 'gini', 
'max\_depth': None, 
'min\_samples\_leaf': 5\}
Best Avarage Score: 0.79746835443

    \end{Verbatim}

    \section{Summury of Results}\label{summury-of-results}

This paper contribution was centered in an automated way to tune
paramenters through GridSearch, rather than doing manually and traying
to guess best combinations.

\paragraph{Analysis with Reduced
DataSet:}\label{analysis-with-reduced-dataset}

With regards to performance, all of experiments were fast, since dataset
was reduced to 1\%. GridSearch presented best parameter
(`min\_samples\_split': 2, `max\_leaf\_nodes': None, `criterion':
`gini', `max\_depth': None, `min\_samples\_leaf': 5) for Decison tree
with a bad accuracy: around 0.80379. GaussianNB was the only classifier
to work without paramenters tune in GridSearch, however, it presented
the best accurace: around 0.943117178612 GridSearch presented best
parameter (kernel:rbf, C:1000, gamma: 0.001) for SVM scoring: 0.91772 In
this scenario, GaussianNB was the simplest and most efficient.

\paragraph{Analysis with Full
DataSet:}\label{analysis-with-full-dataset}

SVM experiment with combination of params and full dataset either was
still running after more than two hours or had the screen fronzen and
did not display any output in more than one attempt. It usually works
very well in complicated domains but it doesn't perform well in very
large datasets, for it can become slow and prone to overfitting. At this
time, results could not be recorded. GaussianNB presented a impressive
performance in the large dataset, and within less than 8 seconds return
the high score of 0.973265073948. That was somehow expected, since it is
not working with combinations of parameters. But still, it is easy to
implement, efficient and performatic, but it can break for some phrases
for considering the words individually. Decision Tree experiment was
with the combination of params in a full dataset reached the best score
0.979393173198. On the other hand, it was slow and it took almost one
hour to display the results.

\textbf{Conclusion:} According to these experiemtns, results suggests
that GaussianNB (Naive Bayes) and Decision Tree would be great options
to work with larger datasets and meet good accuarcy. GaussianNB was much
superior with regards to performance, but Decision Tree had a slight
better accuracy.

And due to the lack of performance, SVM is allegedly not a good option
for working with a large dataset like the one presented in this paper.


    % Add a bibliography block to the postdoc
    
    
\bibliographystyle{plain}
\bibliography{ref}

    \end{multicols}
    \end{document}
